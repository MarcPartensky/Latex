\documentclass{article}

\usepackage{multicol}
\usepackage[a4paper, portrait, margin=1in]{geometry}% \usepackage{color}
\setlength{\columnseprule}{1pt}

\begin{document}
\begin{document}
\title{\vspace{-3cm}Chapitre: Cinématique}
\author{Marc Partensky}
\begin{document}
\maketitle

\section{Rappels Généraux}

\subsection{Rappels}

Solides indéformables \Leftrightarrow \forall A et B \in S
\left\|{\overrightarrow{AB}}\right\| = cst \\

On associera un repère à chaque solide

\subsection{Définitions}

Soit A apparetenant à un solide S1 en mouvement par rapport à un repère R_{O}(\overrightarrow{x}, \overrightarrow{y}, \overrightarrow{z}): \\

$$* \overrightarrow{V_{A\inS/R_{O}}} = (\frac{d\overrightarrow{OA}}{dt})_{R_{0}} $$\\

car O est fixe, dans R_{O} et A appartenant à S_{1} \\

$$* \overrightarrow{\Gamma_{A\in{S_{1}/R_{0}}}} = (\frac{d}{dt}\overrightarrow{V_{A\in{S_{1}/R_{0}}}})_{R_{0}}$$ \\

et par extension, \\

$$ \overrightarrow{\Gamma_{A\in{S_{1}/R_{0}}}} = (\frac{d^2\overrightarrow{OA}}{dt^2})_{R_{0}}

l'accélération est en ms^2\\

\section{Dérivation vectorielle}

Soit \overrightarrow{u} de R_{0}(\overrightarrow{x}, \overrightarrow{y}, \overrightarrow{z}) (repère orthonormé direct)\\

$$ (\frac{d\vec{u}}{dt})_{R_{1}} = (\frac{d\vec{u}}{dt})_{R_{0}} = \overrightarrow{\Omega_{0/1}}\wedge\vec{u}

Comme: \vec{u} \in R_{0},
$$(\frac{d\vec{u}}{dt})_{R_{0}} = \vec{0}$$

avec \overrightarrow{\Omega_{0/1}} Vecteur taux de rotation en rad/s \\
\rightarrow il a pour norme la dérivée de la position angulaire entre les repères 0 et 1.\\
\rightarrow il est parallèle à l'axe autour duquel le repère R_{0} tourne autour du repère R_{1}.\\
\rightarrow il a pour sens le repère (\vec{x_{1}}, \vec{x_{0}}, \overrightarrow{\Omega_{0/1}}), soit direct.









\end{document}
