\documentclass{article}
\usepackage{multicol}
\usepackage[a4paper, portrait, margin=1in]{geometry}% \usepackage{color}
\setlength{\columnseprule}{1pt}
% \def\columnseprulecolor{\color{blue}}

\begin{document}
\title{\vspace{-3cm}Chapitre: Théorie des mécanismes}
\author{Marc Partensky}
\begin{document}
\maketitle

\section*{Objectifs:}
Vérifier qu'un système est apte à réaliser les modifications de mouvements et/ou d'efforts voulues. \\

Il s'agira donc: \\
\begin{itemize}
  \item de rechercher combien de paramètres sont nécessaires pour connnaître la position de tous les solides = MC (mobilité statique)
  \item de rechercher les lois d'entrées-sorties
  \item connaissant les efforts extérieures, peut-on déterminer  toutes les inconnues de liaison:
  \begin{itemize}
    \item si oui, le système est isostatique.
    \item si non, le systeme est dit hyperstatique.
    \item Quelle est l'incidence de l'hyperstatisme sur la géométrie du système? Comment modifier le système pour le rendre isostatique?
    \ldots
  \end{itemize}
\end{itemize}



\section*{Hypothèses:}
\begin{itemize}
  \item Tous les solides sont indéformables.
  \item Toutes les liaisons sont parfaites ( pas de jeu et pas de frottements.)
  \item Ces efforts dynamiques seronts négligés de telle sorte qu'on puisse appliquer le PFS.
\end{itemize}

\section{Graphe de structure}

Rappel: Il y a autant de sommet que de solides dans le système et chaque sommet est relié aux autres s'il y'a une liaison.\\

\underline{ex:}\\
Image \\
p = 6 sommets\\
p = 6 solides\\
p le nombre de solides\\
et NL = Nombre de liaisons du système, ici NL=8\\

Nombre cyclomatique: nombre de boucles indépendantes.\\
$$\mu = NL-(p-1) = NL-p+1$$\\

Dans l'exemple: \mu = 8-6-1 \Rightarrow boucles indépendantes.\\

Image\\

\section{Liaisons équivalentes}
\subsection{liaisons équivalentes en //}

Image \Leftrightarrow Image

\subsubsection*{Méthode cinématique:}
$$\{\nu_{2/1}^{L_{eq}}\} = \{\nu_{2/1}^{L_{1}}\} = \{\nu_{2/1}^{L_{2}}\} = ... = \{\nu_{2/1}^{L_{n}}\}$$ \\
Tous les torseurs doivent être écrits au même point.

\subsubsection*{Méthode statique}
$$\{\tau_{2/1}^{L_{eq}}\} = \{\tau_{2/1}^{L_{1}}\} = \{\tau_{2/1}^{L_{2}}\} = ... = \{\tau_{2/1}^{L_{n}}\}$$ \\
Tous les torseurs doivent être écrits au même point.

\subsection{Liaison en série:}

Image \Leftrightarrow Image

\subsubsection{Méthode statique:}

\begin{multicols}{2}
$$\{\nu_{2/1}^{L_{eq}}\} = \{\nu_{2/1}^{L_{1}}\} = \{\nu_{2/1}^{L_{2}}\} = ... = \{\nu_{2/1}^{L_{n}}\}$$ \\
Tous les torseurs sont écrits au même point. \\
Composition des vitesses.

\columnbreak

$$\{\tau_{2/1}^{L_{eq}}\} = \{\tau_{2/1}^{L_{1}}\} = \{\tau_{2/1}^{L_{2}}\} = ... = \{\tau_{2/1}^{L_{n}}\}$$ \\
Tous les torseurs sont écrits au même point. \\
\end{multicols}
















\end{document}
